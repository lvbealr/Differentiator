\documentclass{article}
\usepackage[english, russian]{babel}
\usepackage[a4paper,top=2cm,bottom=2cm,left=1cm,right=1cm,marginparwidth=1.75cm]{geometry}
\usepackage{amsmath}
\usepackage{amsfonts}
\usepackage{amssymb}
\usepackage{graphicx}
\usepackage[colorlinks=true, allcolors=blue]{hyperref}
\usepackage{setspace}
\usepackage{dashbox}
\begin{document}
\begin{Large}
\begin{onehalfspace}
\begin{center}
\section*{\huge ЭКЗАМЕНАЦИОННАЯ РАБОТА}
\subsection*{\large Дисциплина \framebox{Введение в математический анализ}}
\subsection*{Курс \framebox{1} Семестр \framebox{1} 2024-2025 учебный год}
\subsection*{Фамилия и имя студента \underline{Зубаха Максим}\ \hspace{2cm} № группы \underline{Б05-431}}
\begin{tabular}{|c|c|}
\hline 
Сумма баллов & \hspace{8cm} \\ 
\hline 
Оценка & \hspace{8cm} \\ 
\hline 
\end{tabular} 
\vspace{1cm} 
\hline 
\end{center} 
\begin{large} 
\begin{enumerate} 
\item Найти $y^{(n)}$ при $n \geq 3$. $$y = (x^2 - x + 1)\ln(7-2x)$$ 
\hline 
\item Разложить функцию по формуле Тейлора при $x \rightarrow -1$ до $o((x + 1)^{2n+1})$. $$y = (3x^2 + 6x + 5)e^{2x^2+4x-1}$$ 
\hline 
\item Найти предел. $$\lim\limits_{x \rightarrow 0}(\frac{3}{2} \cdot \frac{\sin{x^2} - \sin^2 x}{e^{x^2} - 1 - x^2})^{\frac{1}{\ch x - 1}}$$ 
\hline 
\item Вычислить производную, ответ упростить. 
$$ f(x) = \cos(x^{3})^{\ln(2 \cdot x)}$$\ \hline \
\end{enumerate}
\end{large}
Уважаемый экзаменатор! Я бедный студент физтеха, который поступил сюда по ЕГЭ. Я сплю по 2 часа в сутки, в моей комнате уже накопилась огромная коллекция энергосов, которой бы хватило всем детям Африки на 20 лет. Все мои соседи говорят, что у меня проблемы с головой. Это странно, ведь соседей то у меня и нет. Один давно отчислился и ушел в армию, другой перевелся во ВШЭ, потому что решил, что он из этих, ну поняли. Сейчас у меня 3 пересдачи, я ботаю деда и матан 24/7. Товарищ экзаменатор, я не умею ничего, кроме как дифференцировать крокодилов. Сейчас я покажу вам все, чему научился за эти мучительные 4 месяца на физтехе... Разъебу этот 4-ый номер!
\section*{Имеем}$$ f(x) = \cos(x^{3})^{\ln(2 \cdot x)}$$\\
\section*{это все, что я имею, физтешку пока себе не нашел :(( \\ Ладно, погнали}\section*{Когда меня спрашивают, что для меня на втором месте после родителей, я говорю что}
$$(2)' = 0$$
\section*{Да, мне было страшно... Но я это сделал:}
$$(x)' = 1$$
\section*{Нам нужно строгое доказательство, поэтому}
$$(2 \cdot x)' = 0 \cdot x + 2 \cdot 1$$
\section*{Что это за пиздец такой? Ну как это может быть в 21м веке?}
$$(\ln(2 \cdot x))' = \frac{0 \cdot x + 2 \cdot 1}{2 \cdot x}$$
\section*{Мама сто раз говорила мне, что}
$$(x)' = 1$$
\section*{Мама сто раз говорила мне, что}
$$(x^{3})' = 3 \cdot x^{2} \cdot 1$$
\section*{Коллеги, прочувствуйте...}
$$(\cos(x^{3}))' = 3 \cdot x^{2} \cdot 1 \cdot -1 \cdot \sin(x^{3})$$
\section*{Любой советский эмбрион знает, что}
$$(\cos(x^{3})^{\ln(2 \cdot x)})' = e^{\ln(2 \cdot x) \cdot \ln(\cos(x^{3}))} \cdot \frac{0 \cdot x + 2 \cdot 1}{2 \cdot x} \cdot \ln(\cos(x^{3})) + \frac{3 \cdot x^{2} \cdot 1 \cdot -1 \cdot \sin(x^{3})}{\cos(x^{3})} \cdot \ln(2 \cdot x)$$
\hline \vspace{1cm} \\
\section*{Хуясе крокодил! Надо бы преобразовать}\\
$$ f'(x) = e^{\ln(2 \cdot x) \cdot \ln(\cos(x^{3}))} \cdot \frac{0 \cdot x + 2 \cdot 1}{2 \cdot x} \cdot \ln(\cos(x^{3})) + \frac{3 \cdot x^{2} \cdot 1 \cdot -1 \cdot \sin(x^{3})}{\cos(x^{3})} \cdot \ln(2 \cdot x)$$\\
\textbf{После несложных упрощений получаем: }\\
$$ f'(x) = e^{\ln(2 \cdot x) \cdot \ln(\cos(x^{3}))} \cdot \frac{2}{2 \cdot x} \cdot \ln(\cos(x^{3})) + \frac{3 \cdot x^{2} \cdot -1 \cdot \sin(x^{3})}{\cos(x^{3})} \cdot \ln(2 \cdot x)$$\\
\end{onehalfspace}
\end{Large}
\end{document}
